\documentclass [t]{beamer}

\usepackage [T1]{fontenc}
\usepackage {fontspec,xunicode,xltxtra}
\usepackage {graphicx}
\usepackage {amsmath, amsfonts, amssymb}
\usepackage {tikz}
\usepackage {array}

\usepackage {color}
\definecolor {rot}{rgb}{1,0,0}
\definecolor {blau}{rgb}{0,0,1}
\definecolor {gruen}{rgb}{0,0.6,0.1}
\definecolor {orange}{rgb}{1,0.3,0}

\usetheme {Boadilla}
\usecolortheme {seahorse}
\setbeamertemplate {navigation symbols}{}
% zeige keine Seitenzahlen an; outhertheme-infolines entnommen und modifiziert
\setbeamertemplate{footline}
{
  \leavevmode%
  \hbox{%
  \begin{beamercolorbox}[wd=.333333\paperwidth,ht=2.25ex,dp=1ex,center]{author in head/foot}%
    \usebeamerfont{author in head/foot}\insertshortauthor~~(\insertshortinstitute)
  \end{beamercolorbox}%
  \begin{beamercolorbox}[wd=.333333\paperwidth,ht=2.25ex,dp=1ex,center]{title in head/foot}%
    \usebeamerfont{title in head/foot}\insertshorttitle
  \end{beamercolorbox}%
  \begin{beamercolorbox}[wd=.333333\paperwidth,ht=2.25ex,dp=1ex,right]{date in head/foot}%
    \hfill\usebeamerfont{date in head/foot}\insertshortdate{}\hfill\insertframenumber{}
%    \insertframenumber{} / \inserttotalframenumber\hspace*{2ex}
  \end{beamercolorbox}}%
  \vskip0pt%
}

\title [GNU Guix packaging]{GNU Guix: Package without a scheme!}
\date [GHM 2013]{GNU Hackers Meeting \\
Paris, 23 August 2013}
\author {Andreas Enge}
\institute [GNU Guix]{\includegraphics [width=2.5cm]{guix-logo.png} \\
\vspace {5mm}
\url {andreas@enge.fr}}


\begin {document}

\begin {frame}
\maketitle
\end {frame}

\begin {frame}{Guix system}
Two chunks of code
\begin {itemize}
\item
Guix package manager
\begin {itemize}
\item
inside \texttt {guix/}
\item
\textcolor {orange}{lots of Scheme}
\end {itemize}
\item
GNU system
\begin {itemize}
\item
inside \texttt {gnu/}
\item
Bootable system (\texttt {gnu/system/}), not yet written
\item
Packages (\texttt {gnu/packages/}),
\textcolor {gruen}{almost no Scheme!}
\end {itemize}
\end {itemize}
\end {frame}


\begin {frame}{Available packages}
\begin {itemize}
\item
\texttt {guix package -A | wc}

454
\item
115 GNU packages (out of 364)
\item
GCC, 6 Schemes, Perl, Python, Ocaml
\item
TeX Live
\item
X.org (client side)
\item
Glib, Cairo, Pango, Gtk+
\item
\textcolor {orange}{Only \textcolor {rot}{free} software!}
\item
\textcolor {gruen}{As pristine as possible (minimal patches)}
\end {itemize}

\begin {center}
\color {blau}
\fbox {\url {http://www.gnu.org/software/guix/package-list.html}}
\end {center}
\end {frame}


\begin {frame}{Package description}
\begin {itemize}
\item
name: Lower case original package name
\item
version
\item
source
\begin {itemize}
\item
uri: Possibility of \texttt {mirror://xxx/...} with \texttt {xxx} $\in$

\texttt {gnu, sourceforge, savannah, cpan, kernel.org, gnome, apache, xorg, imagemagick}
\item
sha256: Hash of tarball
\item
build-system:
\texttt {gnu-build-system, trivial-build-system, python-build-system, perl-build-system,
cmake-build-system}
\end {itemize}
\item
synopsis: As in GNU Womb, \textcolor {orange}{not repeating the package name}
\item
description: Usually taken from web page, two spaces at end of sentence
\item
\textcolor {orange}{license}; see \texttt {guix/licenses.scm}
\item
home-page
\end {itemize}
\end {frame}


\begin{frame}[fragile]{Guix architecture}{Courtesy of Ludovic Courtès}
\begin {center}
  \begin{tikzpicture}[tools/.style = {
                        text width=2.5cm,
                        draw=black, very thick, dashed,
                        rounded corners=2mm
                      },
                      daemon/.style = {
                        rectangle, text width=6.5cm,
                        draw=black, very thick, dashed,
                        rounded corners=2mm, minimum height=15mm
                      },
                      builders/.style = {
                        draw=black, very thick, dashed,
                        text width=4.5cm,
                        rounded corners=2mm,
                      },
                      builder/.style = {
                        draw=black, thick, rectangle,
                        rotate=90
                      }]
    \matrix[row sep=0.8cm, column sep=1cm] {
      \node(tools)[tools]{
          \centerline {\large{\textbf{User tools}}} \\
          \texttt{guix build} \\
          \texttt{guix package} \\
          \texttt{guix gc}}; &
      \node(builders)[builders, text height=4cm]{}
          node at (0, 1.5) {\large{\textbf{Build processes}}}
          node at (0, 1) {chroot, separate UID}
          node[builder] at (-1,-0.5) {make, gcc, \ldots}
          node[builder] at ( 0,-0.5) {make, gcc, \ldots}
          node[builder] at ( 1,-0.5) {make, gcc, \ldots};
      \\
      &
      \node(daemon)[daemon]{\centerline {\large{\textbf{Nix build daemon}}} \\
         \texttt {sudo guix-daemon --build-users-group=guix-builder}};
      \\
    };

    \path[very thick]
      (tools) edge [out=-90, in=180, ->] node[below, sloped]{RPC} (daemon.west);
    \path[->, very thick]
      (daemon) edge (builders);
  \end{tikzpicture}
\end {center}
\end{frame}


\begin {frame}{Let's scheme, at least a little bit!}
\begin {itemize}
\item
Scheme is functional: There are \textcolor {blau}{functions} \ldots

\begin {tabular}{>{\tt}p{4.5cm} >{\tt}p{5.5cm}}
2 + 3; &
\textcolor {rot}{(+} 2 3\textcolor {rot}{)}
\end {tabular}
\item
\ldots\ and \textcolor {blau}{user-defined functions} \ldots

\begin {tabular}{>{\tt}p{4.5cm} >{\tt}p{5.5cm}}
int f (int x, int y) & (\textcolor {rot}{lambda} (x y) (+ x y)) \\
\quad return x+y;    & \\
f (2, 3);            & \textcolor {rot}{(}(lambda (x y) (+ x y)) 2 3\textcolor {rot}{)}
\end {tabular}

\item
\ldots\ but also \textcolor {blau}{global variables} \ldots

\begin {tabular}{>{\tt}p{4.5cm} >{\tt}p{5.5cm}}
a = 5; &
(\textcolor {rot}{define} a 5) \\
2 * a; &
(* 2 a)
\end {tabular}
\item
\ldots\ that can hold functions

\begin {tabular}{>{\tt}p{4.5cm} >{\tt}p{5.5cm}}
int f (int x, int y) & (\textcolor {rot}{define (}f x y\textcolor {rot}{)} (+ x y)) \\
\quad return x+y;    & \\
f (2, 3);            & (f 2 3)
\end {tabular}
\end {itemize}
\end {frame}


\begin {frame}{Let's list and quote}
\begin {itemize}
\item
\textcolor {blau}{Linked lists}

{
\tt
(define a 1) (define b 2) (define c 3)
(\textcolor {rot}{list} a b c)

$\Rightarrow$ (1 2 3)
}
\item
Constant lists / \textcolor {blau}{quote}

{
\tt
'(1 2 3)

$\Rightarrow$ (1 2 3)

(define a 1) (define b 2) (define c 3)
\textcolor {rot}{'}(a b c)

$\Rightarrow$ (a b c)
}
\item \textcolor {blau}{Quasiquote}

{
\tt
(define a 1) (define b 2) (define c 3)
\textcolor {rot}{`}(a b c)

$\Rightarrow$ (a b c)
}
\item \textcolor {blau}{Unquote}

{
\tt
(define a 1) (define b 2) (define c 3)
`(\textcolor {rot}{,}a b \textcolor {rot}{,}c)

$\Rightarrow$ (1 b 3)
}
\end {itemize}
\end {frame}


\begin {frame}{Let's let}
\begin {itemize}
\item
\textcolor {blau}{Local variables}

{
\tt
(define x 10)

(\textcolor {rot}{let} ((x 2) (y 3)) (list x y))

$\Rightarrow$ (2 3)

x

$\Rightarrow$ 10

y

$\Rightarrow$ ERROR: Unbound variable: y
}
\item
\textcolor {rot}{Let's stop!}
\begin {itemize}
\item
H. Abelson, G. Sussman, J. Sussman: \textcolor {blau}{Structure and Interpretation
of Computer Programs}, MIT Press, 2nd ed. 1996,
\url {http://mitpress.mit.edu/sicp/}
\item
D. Sitaram: \textcolor {blau}{Teach Yourself Scheme in Fixnum Days},
\url {http://www.ccs.neu.edu/home/dorai/t-y-scheme/t-y-scheme.html}
\end {itemize}
\end {itemize}
\end {frame}


\begin {frame}{A simple GNU package: units}
\begin {itemize}
\item
\textcolor {blau}{Copy} \texttt {gnu/packages/indent.scm} into
a new source file.
\pause
\item
Modify \textcolor {blau}{copyright},
\textcolor {blau}{module name},
\textcolor {blau}{name} (twice),
\textcolor {blau}{version},
\textcolor {blau}{uri},
\textcolor {blau}{description}.
\item
Add module to \textcolor {blau}{\texttt {gnu-system.am}}.
\pause
\item
\textcolor {blau}{Download} tarball:

\texttt{guix download mirror://gnu/units/units-2.01.tar.gz}
\pause
\item
Modify \textcolor {blau}{hash} and \textcolor {blau}{license}.
\pause
\item
\textcolor {blau}{Check gpg signature.}
\pause
\item
Modify \textcolor {blau}{synopsis}:

\texttt {make sync-synoses}
\pause
\item
\textcolor {blau}{Build}:

\texttt {./pre-inst-env guix build units -K}
\pause
\item
Optionally: Cross-build

\texttt {./pre-inst-env guix build units --target=mips64el-linux-gnuabi64}
\end {itemize}
\end {frame}


\begin {frame}{Update a GNU package}
\begin {itemize}
\item
\textcolor {blau}{Check} for new version:

\texttt {./pre-inst-env guix refresh units}
\item
\textcolor {blau}{Update} package to new version:

\texttt {./pre-inst-env guix refresh units -u}
\end {itemize}
\end {frame}


\begin {frame}{A GNU package with inputs: pspp}
\begin {itemize}
\item
\textcolor {blau}{Copy-paste} a package and modify as before,
try to build.
\pause
\item
Add \textcolor {blau}{inputs}, \textcolor {blau}{native-inputs},
\textcolor {blau}{propagated-inputs}.
\item
Add \textcolor {blau}{configure flags}, try to build.
\pause
\item
Work around \textcolor {blau}{problems}.
Here: Add missing input gettext.
Try to build.
\pause
\item
Despair. Write a bug report.
\begin {itemize}
\item
Missing dependency on gnumeric?
\item
Something related to libreoffice?
\item
\texttt {--without-gui} not tested?
\end {itemize}
\item
Solution (thanks to John Darrington): \\
\texttt {libxml2} was not found; add input \texttt {pkg-config}
\end {itemize}
\end {frame}


\begin{frame}[fragile]{A non-GNU package with build problems: gkrellm}
\begin {itemize}
\item
Dependencies according to
\url {http://members.dslextreme.com/users/billw/gkrellm/gkrellm.html}:
gtk, gdk, glib.

Try build without inputs.
\pause
\item
Following INSTALL, drop configure phase:
\begin{verbatim}
(arguments
`(#:phases
   (alist-delete
   'configure
   %standard-phases)))
\end{verbatim}
There is also \texttt {alist-replace}, \texttt {alist-cons-before},
\texttt {alist-cons-after}.
\end {itemize}
\end{frame}


\begin{frame}[fragile]{A non-GNU package with build problems: gkrellm}
\begin {itemize}
\item
We need to install into store, not \texttt {/usr/local}; \\
and use \texttt {gcc} instead of \texttt {cc}.
\begin{verbatim}
#:make-flags
(let ((out (assoc-ref %outputs "out")))
  (list (string-append "INSTALLROOT=" out)
        "CC=gcc"))
\end{verbatim}
\pause
\item
Add inputs: gettext, gtk+, libsm, libice; native-inputs pkg-config
\pause
\item
Missing \texttt {-lgmodule-2.0} at final link. \\
Solution here: Additional make flag, see \texttt {src/Makefile} \\
\texttt {X11\_LIBS = -lX11 -lSM -lICE -lgmodule-2.0}
\pause
\item
No check phase:

\texttt {\#:tests? \#f}

Sometimes: \texttt {\#:test-target "test"}
\pause
\item
\textcolor {gruen}{Success!}
\end {itemize}
\end{frame}


\begin {frame}{Your turn!}
\begin {itemize}
\item
GNUnet
\item
Improvements to python build system
\item
Firefox
\item
GNOME, and other Gtk+ applications
\item
Qt (4.8 and 5.1)
\item
KDE
\item
OpenJDK
\item
LibreOffice
\item
Fonts
\item
Games
\item
\textcolor {gruen}{Your favourite \textcolor {rot}{free} package!}
\end {itemize}

Subscribe to \texttt {guix-devel@gnu.org}.

Send your patches: \\
\qquad \texttt {git format-patch}
\end {frame}


\begin {frame}
\small
Copyright \\
© 2013 Andreas Enge \texttt {andreas@enge.fr}

Copyright of diagram on page 5 \\
© 2013 Ludovic Courtès \texttt {ludo@gnu.org},
Andreas Enge \texttt {andreas@enge.fr}

Copyright of GNU Guix logo on title page \\
© 2013 Nikita Karetnikov \texttt {nikita@karetnikov.org}

\vspace {3mm}
This work is licensed under the \textcolor {blau}{Creative Commons
   Attribution-Share Alike 3.0} License. To view a copy of this
license, visit
\url{http://creativecommons.org/licenses/by-sa/3.0/} or send a
letter to Creative Commons, 171 Second Street, Suite 300, San
Francisco, California, 94105, USA.

\vspace {2mm}
At your option, you may instead copy, distribute and/or modify
this document under the terms of the \textcolor {blau}{GNU Free Documentation
   License, Version 1.3 or any later version} published by the Free
Software Foundation; with no Invariant Sections, no Front-Cover
Texts, and no Back-Cover Texts. A copy of the license is
available at \url{http://www.gnu.org/licenses/gfdl.html}.

\vspace {2mm}
% Give a link to the 'Transparent Copy', as per Section 3 of the GFDL.
The source of this document is available from
\url{http://git.sv.gnu.org/cgit/guix/maintenance.git}.
\end {frame}

\end {document}
