% Compile with xelatex.

\documentclass [11pt]{article}

\usepackage [a4paper,vscale=0.75]{geometry}
\usepackage {fontspec}
\usepackage {xunicode}

\renewcommand {\section}[1]{\stepcounter {section}%
{\vspace {1em}\noindent\Large \bf Article \thesection: #1 \par}}


\begin {document}

\textit {
This document is a non-official English translation of the official French
``règlement intérieur'' governing the Guix Europe association; only the
French version is binding.
}

\begin {center}
\bf \huge {Interior Reglementary}
\end {center}

The Interior reglementary has the same binding force towards the members
as the statutes of the Association. This reglementary has been written
in conformance with the process established by the statutes.


\section {Amount of the membership fees}

Membership fees are computed by calendar year; in the case of admission
during the year, the full amount is due.
The current membership fees are 10€ per year.


\section {Conditions of admission}

A person desiring to obtain the member status needs to (in conformance
with the reglementation):
\begin {itemize}
\item
be major or be represented by a major legal guardian;
\item
communicate in written or by e-mail to the Board a demand for admission;
\item
accompany this demand by:
\begin {itemize}
\item
their name and first name;
\item
their birthdate;
\item
their complete postal address, their e-mail address and any other
means of communication by which to join them;
\item
potentially their OpenPGP key;
\item
potentially their motivations to join the Association.
\end {itemize}
\end {itemize}

The membership demand must be accompanied by the membership fees.
These may be paid in cash, by wire transfer to the account of the
Association or by check.

In conformance with the statutes, the Solidary Administrative Council
reserves the right to accept or not a new member.


\section {Presence in electronic form}

The Board may make possible a participation to the Solidary Administrative
Council and to the General Assemblies in electronic form, be it by
audio- or videoconference, e-mail, social networks or any other form
of electronic participation.
Members participating in electronic form are counted as present just
as members who are physically present, in particular as quorums are
concerned.


\section {Electronic vote}

The Board may, additionally to and complementing the vote in
physical presence, make an electronic vote possible for the meetings
of the Solidary Administrative Council and the General Assemblies,
by any suitable means.

In particular, a member may vote sending to a member of the Board an
e-mail signed with their OpenPGP key or, if such a key has not been
specified during their admission, coming from their e-mail address
communicated during their admission.

Members taking part in an electronic vote are counted as being present
for the topics covered by this vote with the same rights as members
who are physically present, in particular as quorums are concerned.


\end {document}
