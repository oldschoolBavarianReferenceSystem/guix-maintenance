% Compile with xelatex.

\documentclass [11pt]{article}

\usepackage [a4paper,vscale=0.75]{geometry}
\usepackage {fontspec}
\usepackage {xunicode}
\usepackage [francais]{babel}

\renewcommand {\section}[1]{\stepcounter {section}%
{\vspace {1em}\noindent\Large \bf Article \thesection\ : #1 \par}}


\begin {document}

\begin {center}
\bf \huge {Règlement intérieur}
\end {center}

Le Règlement intérieur a la même force obligatoire pour les membres que les
statuts de l'Association. Ce règlement a été élaboré conformément au
processus établi par les statuts.


\section {Montant de la cotisation}

Le montant de l'adhésion est calculé par année calendaire; lors d'une
adhésion en cours d'année, le montant complet est dû.
À ce jour, la cotisation s'élève à 10€ par an.


\section {Conditions d'admission}

La personne désirant obtenir le statut d'adhérent devra (conformément à la
règlementation) :
\begin {itemize}
\item
être majeure ou représentée par une personne majeure en ayant la
responsabilité;
\item
communiquer par écrit ou par courriel une demande d'adhésion au Bureau;
\item
accompagner cette demande de:
\begin {itemize}
\item
ses nom et prénoms;
\item
sa date de naissance;
\item
son adresse postale complète, son adresse mél et tout autre moyen de
communication permettant de la joindre;
\item
éventuellement sa clef OpenPGP;
\item
éventuellement, les motivations qui la poussent à rejoindre l'Association.
\end {itemize}
\end {itemize}

La demande d'adhésion doit être accompagnée du règlement de la cotisation.
Ce règlement peut être effectué en espèces, par virement bancaire sur le
compte d'Association ou par chèque.

Conformément aux statuts, le Collège d'Administration Solidaire se réserve
le droit d'accepter ou non un nouveau membre.


\section {Présence sous forme électronique}

Le Bureau peut rendre possible la participation aux Collège d'Administration
Solidaire et aux Assemblées Générales sous forme électronique, que ce soit
par audio- ou visioconférence, messagerie électronique, réseaux sociaux ou
toute autre forme de participation électronique.
Les membres particant de façon électronique sont comptabilisés présents
au même titre que les membres physiquement présents, notamment en ce qui
concerne les quorums.


\section {Vote électronique}

Le Bureau peut, en sus et en complément du vote en présence physique,
permettre un vote électronique aux sessions du Collège d'Administration
Solidaire et des Assemblées Générales, par tous les moyens utiles.

En particulier, un membre peut voter en evoyant à un membre du Bureau un
courriel signé par sa clef OpenPGP ou, si une telle clef n'a pas été
spécifiée lors de son inscription, provenant de son adresse mél communiqué
lors de son inscription.

Les membres participant à un vote électronique sont comptabilisés présents
pour les sujets concernés par ce vote au même titre que les membres
physiquement présents, notamment en ce qui concerne les quorums.


\end {document}
