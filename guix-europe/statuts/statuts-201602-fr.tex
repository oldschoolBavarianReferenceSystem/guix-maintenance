% Compile with xelatex.

\documentclass [11pt]{article}

\usepackage [a4paper,vscale=0.75]{geometry}
\usepackage {fontspec}
\usepackage {xunicode}
\usepackage [francais]{babel}

\renewcommand {\section}[1]{\stepcounter {section}%
{\vspace {1em}\noindent\Large \bf Article \thesection\ : #1 \par}}


\begin {document}

\begin {center}
\bf \huge {Statuts}
\end {center}


\section {Titre de l'Association}

Il est fondé entre les adhérents aux présents statuts une association
régie par la loi du 1er juillet 1901 et le décret du 16 août 1901, ayant
pour titre: Guix Europe.


\section {But de l'Association}

L'Association a pour but la promotion, l'utilisation et le développement
de logiciels et de systèmes d'exploitation respectueux des libertés des
usagers, en favorisant en particulier les utilisations à des fins de
recherche ou d'éducation sans volonté commerciale.
L'Association s'engage notamment à soutenir le projet GNU Guix.


\section {Siège social}

Le siège social de l'Association est fixé à Bordeaux.
Il pourra être transféré sur décision du Collège d'Administration Solidaire.


\section {Membres de l'Association}

L'Association se compose exclusivement de membres actifs,
personnes physiques ou morales s'acquittant d'une cotisation fixée
annuellement par l'Assemblée Générale.
Ils sont membres de l'Assemblée Générale avec voix délibérative.


\section {Admission}

Pour faire partie de l'Association, il faut être agréé par le
Collège d'Administration Solidaire qui statue sur les demandes d'admission
présentées, adhérer aux présents statuts et s'acquitter de la cotisation
dont le montant est fixé par l'Assemblée Générale.


\section {Radiation}

La qualité de membre se perd par:
\begin {itemize}
\item
démission;
\item
le décès de la personne physique ou la dissolution de la personne morale;
\item
la radiation prononcée par le Collège d'Administration Solidaire pour
non-paie\-ment de la cotisation, pour infraction aux statuts, pour motif
portant préjudice aux intérêts moraux et matériels de l'Association, ou pour
motif grave.

Le règlement intérieur pourra préciser quels sont les motifs graves.
\end {itemize}


\section {Ressources}

Les ressources de l'Association comprennent:
\begin {itemize}
\item
le montant des cotisations;
\item
les subventions de l'Union européenne, de l'État, des régions, des
départements et des communes, ou de tout autre organisme public;
\item
les sommes perçues en contrepartie des prestations fournies par
l'Association;
\item
toutes les autres ressources autorisées par les textes législatifs ou
réglementaires.
\end {itemize}


\newpage
\section {Le Bureau}

Le Bureau assure le bon fonctionnement de l'Association sous le contrôle
du Collège d'Administration Solidaire dont il prépare les réunions.
Il est composé de deux membres de l'Association, une personne assurant la
présidence (appelée «Présidence» dans la suite) et une personne assurant la
trésorerie (appelée «Trésorerie» dans la suite), élues par l'Assemblée
Générale.

La Présidence ou la Trésorerie représentent l'Association dans tous les actes
de la vie civile. Elles ont qualité pour s'exprimer au nom de l'Association
vis à vis de leurs interlocuteurs ou des médias, pour ester en justice
au nom de l'Association ainsi que pour présenter toute réclamation
auprès de toutes administrations, notamment en matière fiscale, et pour
ouvrir tout compte bancaire ou postal. Elles ordonnent les dépenses.
Elles peuvent avec l'accord du Collège d'Administration Solidaire déléguer
leurs pouvoirs à un autre membre du Collège d'Administration Solidaire,
ou plusieurs, sur un thème, un projet, ou vis-à-vis d'un interlocuteur défini.
En cas de représentation en justice, un mandataire peut remplacer
un membre du Bureau par procuration.

La Présidence est chargée de rédiger les procès-verbaux des réunions du
Collège d'Administration Solidaire et de l'Assemblée Générale et de tenir le
registre prévu par la loi. En cas d'empêchement, la Présidence est remplacée
par la Trésorerie, ou par un autre membre du Collège d'Administration Solidaire
désigné par la Trésorerie.

La Trésorerie est chargée de tenir ou de faire tenir sous son contrôle la
comptabilité de l'Association. Elle perçoit les recettes; elle effectue tout
paiement sous réserve de l'accord de la Présidence. En cas d'empêchement, la
Trésorerie est remplacée par la Présidence, ou par un autre membre du Collège
d'Administration Solidaire désigné par la Présidence. Vis-à-vis des organismes
bancaires ou postaux, la Présidence, la Trésorerie ou tout autre membre du
Collège d'Administration Solidaire désigné par la Présidence avec l'accord de
la Trésorerie, ont pouvoir, chacun séparement, de signer tout moyen de paiement
(chèques, virements, etc).


\section {Le Collège d'Administration Solidaire}

L'Association est dirigée par un Collège d'Administration Solidaire qui
comporte d'office les membres du Bureau ainsi que zéro ou plus membres de
l'Association élus par l'Assemblée Générale.

Le Collège d'Administration Solidaire se réunit au moins une fois tous les
ans, sur convocation d'un membre du Bureau ou sur la demande d'au moins d'un
quart de ses membres.

La présence ainsi que le vote d'un membre du Collège d'Administration
Solidaire peut être sous forme électronique, conformément aux dispositions
prévues par le règlement intérieur.

Les décisions sont prises à la majorité absolue des suffrages.

Tout membre du Collège d'Administration Solidaire qui, sans excuse, n'aura
pas assisté à trois réunions consécutives, pourra être considéré comme
démissionnaire.

Le Collège d'Administration Solidaire est investi des pouvoirs les plus
étendus dans les limites de l'objet de l'Association et dans le cadre des
résolutions adoptées par l'Assemblée Générale. Il peut autoriser tous
actes ou opérations qui ne sont pas statutairement de la compétence
de l'Assemblée Générale ordinaire ou extraordinaire.

Il est chargé:
\begin {itemize}
\item
de la mise en œuvre des orientations décidées par l'Assemblée Générale;
\item
de la préparation des bilans, de l'ordre du jour et des propositions
de modification du règlement intérieur présentés à l'Assemblée Générale;
\item
de la préparation des propositions de modifications des statuts présentés
à l'Assemblée Générale extraordinaire. Le Collège d'Administration Solidaire
peut déléguer tel ou tel de ses pouvoirs, pour une durée déterminée, à un ou
plusieurs de ses membres, en conformité avec le règlement intérieur.
\end {itemize}


\section {Les Assemblées Générales}

L'Assemblée Générale Ordinaire ou Extraordinaire comprend tous les membres
de l'Association à jour de leur cotisation de l'année en cours.
Quinze jours au moins avant la date fixée par le Collège d'Administration
Solidaire, les membres de l'Association sont convoqués par les soins de la
Présidence.
L'ordre du jour est indiqué sur la convocation.
L'Assemblée Générale est présidée par la Présidence.

La présence d'un membre à une Assemblée Générale peut être sous forme
électronique conformément aux dispositions prévues par le règlement
intérieur.

Un membre peut se faire représenter par un autre membre en faisant la
demande au Bureau par écrit ou par courriel; un membre peut représenter
au plus un autre membre.


\section {L'Assemblée Générale Ordinaire}

L'Assemblée Générale Ordinaire se réunit obligatoirement une fois par an.
Lors de cette réunion dite «annuelle», la Présidence soumet à l'Assemblée
Générale un rapport sur l'activité de l'Association. La Trésorerie soumet
le rapport financier comportant les comptes de l'exercice écoulé.
Il est ensuite procédé à l'élection des membres du Bureau, puis du
Collège d'Administration Solidaire.

Il est ensuite procédé à l'examen des autres questions figurant à l'ordre
du jour.

Les décisions sont prises à la majorité absolue des suffrages exprimés par
les membres présents ou représentés.

Un vote électronique pourra être rendu possible conformément aux dispositions
prévues par le règlement intérieur.

L'Assemblée Générale Ordinaire peut également être convoquée à tout moment
à la demande de la majorité des membres du Collège d'Administration Solidaire.


\section {L'Assemblée Générale Extraordinaire}

L'Assemblée Générale Extraordinaire se prononce sur les modifications à
apporter aux statuts et sur la dissolution de l'Association. Elle se
réunit à la demande d'un membre du Bureau ou de la majorité des membres
du Collège d'Administration Solidaire. L'Assemblée Générale Extraordinaire
ne peut se prononcer valablement que si les deux-tiers des membres de
l'Association sont présents ou représentés.
Les décisions sont prises à la majorité des deux-tiers des suffrages
exprimés par les membres présents ou représentés.

L'Assemblée Générale Extraordinaire a également la possibilité de prendre
toutes les décisions prévues pour l'Assemblée Générale Ordinaire, et ce
dans les mêmes circonstances, c'est-à-dire sans minimum de représentation
des membres, et à la majorité absolue des suffrages exprimés. Si le quorum
des deux-tiers des membres n'est pas atteint, l'Assemblée sera de fait
une Assemblée Générale Ordinaire, et statuera sur les points de l'ordre
du jour qui le permettent.

Si le quorum n'est pas atteint, l'Assemblée Générale Extraordinaire est
convoquée à nouveau, à quinze jours d'intervalle. Elle peut alors délibérer
quel que soit le nombre de membres présents et représentés, et les
délibérations sont prises à la majorité absolue des membres présents ou
représentés.


\section {Règlement Intérieur}

Un règlement intérieur est établi par le Collège d'Administration Solidaire
qui le fait approuver par l'Assemblée Générale. Ce règlement éventuel est
destiné à fixer les divers points non prévus par les statuts, notamment
ceux qui ont trait à l'administration interne de l'Association.


\section {Dissolution}

En cas de dissolution prononcée par l'Assemblée Générale Extraordinaire, un
ou plusieurs liquidateurs sont nommés par celle-ci. L'actif, s'il y a lieu,
est dévolu par cette Assemblée Générale Extraordinaire à une ou plusieurs
associations ayant un objet similaire ou à tout établissement à but social
ou culturel de son choix.

\end {document}
