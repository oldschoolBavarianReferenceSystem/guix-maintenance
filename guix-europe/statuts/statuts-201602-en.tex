% Compile with xelatex.

\documentclass [11pt]{article}

\usepackage [a4paper,vscale=0.75]{geometry}
\usepackage {fontspec}
\usepackage {xunicode}

\renewcommand {\section}[1]{\stepcounter {section}%
{\vspace {1em}\noindent\Large \bf Article \thesection: #1 \par}}


\begin {document}

\textit {
This document is a non-official English translation of the official French
bylaws governing the Guix Europe association; only the French version is
binding.
}

\begin {center}
\bf \huge {Statutes}
\end {center}


\section {Title of the Association}

It is founded among the persons adhering to the present statutes an
association under the law of the 1st of July, 1901, and the decree of the
16th of August 1901, having as title:
Guix Europe.


\section {Goal of the Association}

The Association has the goal of promoting, using and developing software
and operating systems that respect users' freedoms, supporting in particular
usages in research and education without any commercial goal.
The Association is specially dedicated to supporting the GNU Guix project.


\section {Registered office}

The registered office of the Association is at Bordeaux.
It may be transferred by a decision of the Solidary Administration Council.


\section {Members of the Association}

The Association is formed exclusively of active members,
natural or legal persons paying a membership fee decided yearly by
the General Assembly.
They take part in the General Assembly with voting rights.


\section {Admission}

To become a member of the Association, a person needs to be admitted by
the Solidary Administration Council which decides on the submitted
demands of admittance, adhere to the present statutes and pay the
membership fee the amount of which is fixed by the General Assembly.


\section {Cancellation of membership}

Membership is lost by:
\begin {itemize}
\item
retirement;
\item
demise of the natural person or liquidation of the legal person;
\item
exclusion pronounced by the Solidary Administration Council
due to non-payment of the membership fee, violation of the statutes,
behaviour apt to damage the moral or material interests of the Association,
or for grave reasons.

The internal regulations may give a more precise
definition of grave reasons.
\end {itemize}


\newpage
\section {Ressources}

The ressources of the Association comprise:
\begin {itemize}
\item
the membership fees;
\item
subsidies by the European Union, the State, the regions, the departments
and townships, or any other public entity;
\item
payment received for services rendered by the Association;
\item
any other ressources authorised by the legal or reglementary texts.
\end {itemize}


\section {The Board}

The Board ensures the good functioning of the Association under the control
of the Solidary Administration Council of which it prepares the meetings.
It is composed of two members of the Association, a person ensuring the
presidency (called ``Presidency'' in the following) and a person ensuring the
treasury (called ``Treasury'' in the following), elected by the General
Assembly.

The Presidency and the Treasury represent the Association in all acts of
civil life. They are entitled to express themselves in the name of the
Association towards their interlocutors and the media, to act in court
in the name of the Association as well as to make any opposition towards
any administration, in particular in fiscal matters, and to open a bank
or post account. They mandate expenses.
They may, with the agreement of the Solidary Administrative Council,
delegate their powers to another member or to several members of the
Solidary Administrative Council, on a topic, a project, or towards a
fixed interlocutor. In the case of appearance in court, a proxy may replace
a member of the Board by power of procuration.

The Presidency is in charge of writing the minutes of the meetings of the
Solidary Administrative Council and of the General Assembly and of keeping
the lawful registry. In case of indisposedness, the Presidency is substituted
by the Treasury, or by another member of the Solidary Administrative Council
designated by the Treasury.

The Treasury is in charge of executing or having executed under its control
the accounting of the Association. It treats incoming payments; it executes
all payments under provision of agreement by the Presidency. In case of
indisposedness, it is substituted by the Presidency, or by another member
of the Solidary Administrative Council designated by the Presidency.
Towards banks and the post, the Presidency, the Treasury or any other member
of the Solidary Administrative Council designated by the Presidency with
agreement by the Treasury, have the power, individually, to sign any
instrument of payment (checks, wire transfers, etc).


\section {The Solidary Administrative Council}

The Association is governed by a Solidary Administrative Council,
comprising the members of the Board and zero or more members of the
Association elected by the General Assembly.

The Solidary Administrative Council meets at least once every year
upon invitation by a Board member or by demand of at least a quarter
of its members.

Presence as well as voting by a member of the Solidary Administrative
Council may be in electronic form as stipulated by the interior reglementary.

Decisions are taken with an absolute majority of votes.

Any member of the Solidary Administrative Council that has, without being
excused, not assisted to three consecutive meetings, may be considered
as having resigned.

The Solidary Administrative Committee is invested with the highest powers
within the limits of the goal of the Association and the resolutions
adopted by the General Assembly. It may authorise any activities or
operations that by the statutes do not fall into the sole competence
of the ordinary or extraordinary General Assembly.

It is in charge of:
\begin {itemize}
\item
realising the actions decided by the General Assembly;
\item
preparing the financial statements, the agenda and propositions
for modifications of the interior reglementary presented at the
General Assembly;
\item
preparing the propositions of modifications of the statutes presented
at the extraordinary General Assembly. The Solidary Administrative Council
may delegate any of its powers for a limited duration to one or several
of its members, conforming to the interior reglementary.
\end {itemize}


\section {The General Assemblies}

The Ordinary or Extraordinary General Assembly comprises all members
of the Association having paid their membership fees of the current year.
Fifteen days at least before the date fixed by the Solidary Administrative
Council, the members of the Association are invited by the Presidency.
The agenda is communicated in the invitation.
The Presidency presides the General Assembly.

Presence of a member at a General Assembly may be in electronic form
as stipulated by the interior reglementary.

A member may ask to be represented by another member by writing or sending
an e-mail to the Board; a member may represent at most one other member.


\section {The Ordinary General Assembly}

The Ordinary General Assembly meets at least once a year.
During this meeting said ``annual'', the Presidency presents to
the General Assembly a report on the activity of the Association.
The Treasury prensents a financial report comprising the balance
of the past period.
After that the members of the Board are elected, then those of the
Solidary Administrative Council.

After that the further questions of the agenda are treated.

Decisions are taken with an absolute majority of the expressed votes
by present or represented members.

An electronic vote may be made possible as stipulated in the interior
reglementary.

The Ordinary General Assembly may also be united at any moment upon
the demand of a majority of the members of the Solidary Administrative
Council.


\newpage
\section {The Extraordinary General Assembly}

The Extraordinary General Assembly decides about modifications of the
statutes and about the dissolution of the Association. It is united
upon the demand of a member of the Board or of the majority of the
members of the Solidary Administrative Council. The Extraordinary General
Assembly can only take valid decisions if two thirds of the members
of the Association are present or represented.
Decisions are taken with a two thirds majority of the votes
expressed by the present or represented members.

The Extraordinary General Assembly may also take all decisions in the
competence of the Ordinary General Assembly, under the same circumstances,
that is, without a quorum of representation of its members and with
an absolute majority of the expressed votes. If the quorum of two thirds
of the members is not reached, the Assembly becomes in fact an Ordinary
General Assembly, and decides on the topics of the agenda where it
is competent.

If the quorum is not reached, the Extraordinary General Assembly is
united again fifteen days later. It may then decide regardless
of the number of present and represented members, and the decisions are
taken with an absolute majority of the present or represented members.


\section {Interior Reglementary}

An interior reglementary is established by the Solidary Administrative
Council and approved by the General Assembly. This potential reglementary
is supposed to fix diverse topics not covered by the statutes, in
particular pertaining to the internal administration of the Association.


\section {Dissolution}

In the case of a dissolution pronounced by the Extraordinary General
Assembly, it designates one or more liquidators. Assets, if present,
are transferred by this Extraordinary General Assembly to one or more
associations persuing a similar goal or to any body of its choice
with a social or cultural goal.

\end {document}
